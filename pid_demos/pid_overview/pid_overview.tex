% Page formatting
\documentclass[11pt]{article}
\usepackage{algorithm,algorithmic,amsmath,amsgen,amstext,amssymb,amsthm,amsbsy,amsopn,amsfonts,esint,hyperref}
\usepackage{bm}
\usepackage{tikz}
\usepackage[utf8]{inputenc}
\usepackage[english]{babel}
\usepackage{tabularx}
\pagestyle{plain}

\usetikzlibrary{shapes,arrows.meta,decorations.markings,positioning}

\setlength{\textwidth}{6.5 in}
\setlength{\textheight}{9 truein}
\setlength{\evensidemargin}{ 0 in}
\setlength{\oddsidemargin}{0 in}
\setlength{\topmargin}{-0.5 in}
\setlength{\parskip} {0.125 in}
\setlength{\parindent} {0 in}

\renewcommand{\familydefault}{\sfdefault}

% Remappings
\newcommand{\R}{\mathbb{R}}
\newcommand{\C}{\mathbb{C}}
\renewcommand\qedsymbol{ \blacksquare }
\newcommand{\arrowtext}[1]{\scriptstyle{\boldsymbol{#1}}}

% Theorems
\newtheorem{theorem}{Theorem}[section]
\newtheorem{corollary}{Corollary}[theorem]
\newtheorem{lemma}[theorem]{Lemma}

% Definitions
\theoremstyle{definition}
\newtheorem{definition}{Definition}[section]

% Beginning of document
\begin{document}
\noindent
\begin{minipage}{1.00\textwidth}
    \begin{flushleft}
        % title
        \textbf{\huge PID Overview} \\[1ex]
        % author
        \Large Jeremy Evans
    \end{flushleft}
\noindent\rule{\textwidth}{1pt}
\end{minipage}

% tikz colors
\definecolor{wmugold}{RGB}{241,197,0}

% == tikzset ==
\tikzset{
    block/.style={
        draw,
        fill=wmugold,
        rectangle,
        minimum height=3em,
        minimum width=7em,
    },  
    sum/.style={
        draw,
        fill=wmugold,
        circle,
        node distance=1cm
    },  
    input/.style={coordinate},
    output/.style={coordinate},
    blank/.style={coordinate},
    pinstyle/.style={
        pin edge={to-,thin,black}
    },  
    big_arrow/.style={
        decoration={markings,mark=at position 1 with {\arrow[scale=2,>=latex]{>}}},
        postaction={decorate},
        shorten >=0.4pt
    },  
}

% =======================================================================================================================
% body

% section: Introduction
\section{Introduction}

This document introduces the theory of PID control and its programmatic implementation. We will define
what a PID controller is, what its purpose is, loop tuning, and its implementation in software. It is meant to
serve as an aid for designing the programming logic of a PID controller deployable to an MSP430 MCU.

A working understanding of the Laplace and Z transforms would be helpful in reading this document (if only to 
verify the correctness of the implementation). Nevertheless, we will document the derivation in full (though we may
omit the calculus and algebra in the final progress report).

% section: Overview
\section{Overview}

A PID controller (proportional-integral-derivative controller) is a feedback control loop mechanism that receives a desired  setpoint 
$r(t)$ as input and outputs a \textit{process variable} (PV), denoted as $y(t)$, which is then fed back into the system
as input. It continuously calculates an  error value $e(t)$ as the difference between the desired setpoint and the process variable
and applies a correction based on  proportional $(P)$, integral $(I)$, and derivative $(D)$ terms, which are summed
together to make up the  control variable, $u(t)$ affecting the value of the process variable, hence the name.  
In applying this correction over time, it attempts to stabilize the output, i.e., eliminate the oscillation of the error value
(or achieve marginal stability, or bounded oscillation, though specifications vary between applications).

Corrections are achieved by multiplying the $P$, $I$, and $D$ terms by constants $K_{p}$, $K_{i}$, and $K_{d}$ respectively, each chosen through
a process known as \textit{loop tuning.} Improperly chosen constants would result in the error value diverging (with or without oscillation),
whereas properly chosen constants would have the desired effect of the error value converging to zero (or oscillating within 
an acceptable bounds).

\newpage

% subsection: Mathematical Definition
\subsection{Mathematical Definition}

The parallel form of PID control is mathematically defined as:

\begin{equation} \label{eq1}
    \boxed{
        u(t) = K_{p}e(t) + K_{i}\int_{0}^{t}e(\tau)d\tau + K_{d}\frac{de(t)}{dt}
    }
\end{equation}

where

\begin{itemize}
    \item $u(t)$ is the \textit{control variable}, the parameter that is controlled,
    \item $K_{p} \in \R \geq 0$ is the \textit{proportional gain,}
    \item $K_{i} \in \R \geq 0$ is the \textit{integral gain,}
    \item $K_{d} \in \R \geq 0$ is the \textit{derivative gain,}
    \item $e(t) = r(t) - y(t)$ is the error ($r(t) = SP$ is the \textit{setpoint,} $y(t) = PV$ is the \textit{process variable}),
    \item $t$ is the \textit{instantaneous time} (i.e., the current time),
    \item $\tau \in \R$ are measurements of time in the range $[0, t]$.
\end{itemize}

% subsection: PID Block Diagram
\subsection{PID Block Diagram}

The PID algorithm can be represented by the following block diagram:

% PID block diagram
\begin{tikzpicture}[auto]
    % nodes: input
    \node [input, name=input]{}; 
    \node [sum, right = 1.3cm of input] (sum_1) {$\Sigma$};
    \node [blank, name = root, right = 1.7cm of sum_1]{};

    % nodes: PID
    \node [block, text width = 7em, right = 1cm of root] (pid_integral) {\textbf{I}: $K_{i}\int_{0}^{t}e(\tau)d\tau$};
    \node [block, text width = 7em, above = of pid_integral] (pid_proportional) {\textbf{P}: $K_{p}e(t)$};
    \node [block, text width = 7em, below = of pid_integral] (pid_derivative) {\textbf{D}: $K_{d}\frac{de(t)}{dt}$};

    % nodes: Output
    \node [sum, right = 2cm of pid_integral] (sum_2) {$\Sigma$}; 
    \node [block, right = 1.5cm of sum_2] (plant_process) {Plant Process};
    \node [blank, name=feedback, right = 0.2cm of plant_process]{};
    \node [output, name=output, right = 1.60cm of feedback]{};
    \node [block, below = of pid_derivative] (sensor) {Sensor};

    % arrows: input -> PID
    \draw [big_arrow] (input) -- node [above]{$r(t)$} node [below,pos=0.90] {$\arrowtext{+}$}(sum_1);
    \draw [-] (sum_1) -- node {$e(t)$}(root);
    \draw [big_arrow] (root) -- (pid_integral);
    \draw [big_arrow] (root) |- (pid_proportional);
    \draw [big_arrow] (root) |- (pid_derivative);

    % arrows: PID -> PV
    \draw [big_arrow] (pid_integral) -- node [pos=0.90]{$\arrowtext{+}$}(sum_2);
    \draw [big_arrow] (pid_proportional) -| node [pos=0.90]{$\arrowtext{+}$} (sum_2);
    \draw [big_arrow] (pid_derivative) -| node [pos=0.90]{$\arrowtext{+}$}(sum_2);
    \draw [big_arrow] (sum_2) -- node{$u(t)$}(plant_process);

    % arrows: p(v) -> r(t)
    \draw [-] (plant_process) -- (feedback);
    \draw [-] (feedback) |- (sensor);
    \draw [big_arrow] (sensor) -| node [below,pos=0.30] {measurement} node [pos=0.96]{$\arrowtext{-}$}(sum_1);
    \draw [big_arrow] (feedback) -- node {$y(t)$}(output);
\end{tikzpicture}

Represented in a high-level model of a heat treatment furnace, $r(t),$ the setpoint, would be the desired
furnace temperature; $u(t),$ the control variable, would be the gas flow rate; and $y(t),$ the process
variable, would be the measured furnace temperature.

% subsection: PID Terms
\subsection{PID Terms}

% subsubsection: Proportional
\subsubsection{Proportional}

Given by $P(t) := K_{p}e(t),$ the proportional term produces an output that is proportional to the current error value $e(t);$ the greater
the error value, the greater the control output. This output is multiplied by a gain factor $K_{p} \in \R \geq 0$ that determines how responsive
the controller should be with a given error-value. Large values of $K_{p}$ result in a large change in the output for a given change in the
error. However, exceptionally large values may result in the output \textit{overshooting} the setpoint (exceeding the value of the setpoint)
and, in the worst case, destabilizing the system whereupon the error rate diverges.

One caveat to the proportional term is that a non-zero error is required for it to have any effect; if the error is zero, it will not 
produce a response. Because of this, it generally operates with a \textit{steady-state error,} which is the difference between the desired 
final output and the actual one.

% subsubsection: Integral
\subsubsection{Integral}

Given by $I(t) := K_{i}\int_{0}^{t}e(\tau)d\tau,$ the integral term's output is proportional to both the magnitude and duration
of the error; its integral is the sum of the instantaneous error over time and represents the accumulated offset. Because it sums
the error over time, the integral response will continually increase over time so long as the error is non-zero, effectively driving
the steady-state error inherent in the proportional term to zero.

If the setpoint suddenly changes to a drastic degree or if the measured process variable jumps significantly, \textit{integral windup}
can occur wherein the plant process of a PID controller can saturate; that is, reach its physical limit (e.g., drive a valve to be fully
open or closed). In this case, no physical change in the system can affect the integral term; it will continue to grow without bounds.

Several solutions to this problem are:

\begin{itemize}
    \item Increase the setpoint incrementally (in a suitable ramp)
    \item Initialize the controller integral to a desired value (e.g., before the problem occurs)
    \item Disable the integral function until the process variable has entered the controllable region
\end{itemize}

% subsubsection: Derivative
\subsubsection{Derivative}

Given by $D(t) := K_{d}\frac{de(t)}{dt},$ the output of the derivative term is proportional to the rate of change of the error value
and is intended to decrease the rate at which the error increases. Large values of $K_{d}$ correlate to large responses to changes in the
error. However, it is sensitive to noise in the process variable (moreso for large values of $K_{d}$) and is often multiplied by a low-pass
filter to mitigate the high frequency gain and noise.

% section: Derivation
\section{Derivation}

To implement the PID algorithm in software, we need to transform the equation into a form that can be represented in code. One way to do
this is to derive the transfer function of the PID algorithm in the \textit{s-domain} (where output is viewed with respect to frequency
rather than time), and discretize the transfer function using what is called the \textit{bilinear transform} or \textit{Tustin's method}.
By discretizing the transfer function, we can arrive at a linear difference equation (also known as a recurrence relation), which
can easily be implemented in code.

% subsection: PID Transfer Function
\subsection{PID Transfer Function}

The Laplace transform of a function $f(t)$, defined for all real numbers $t \geq 0$, is the function $F(s)$, which is a unilateral transform defined by 
\begin{equation} \label{eq2}
    \boxed{
        F(s) := \mathcal{L}\{f(t)\} = \int_{0}^{\infty}f(t)e^{-st}dt
    }
\end{equation}

where $s = \sigma + i\omega$ (with real numbers $\sigma$ and $\omega$) is a complex number denoting frequency.

Divide \textbf{eq.~\ref{eq1}} into three functions $P(t),$ $I(t),$ and $D(t),$ where

\begin{align*}
\begin{split}
    P(t) &:= K_{p}e(t) \\
    I(t) &:= K_{i}\int_{0}^{t}e(\tau)d\tau \\
    D(t) &:= K_{d}\frac{de(t)}{dt}
\end{split}
\end{align*}

Recall that the integral of a real-valued function $f(t)$ from $t = 0$ to $t = \infty$ is equal to evaluating the integral
from $0$ to some number $a \in \R$ that approaches $\infty$ from the left:

\begin{align*}
\begin{split}
    \int_{0}^{\infty}f(t)dt = \lim_{a\to\infty^{-}}\left(\int_{0}^{a}f(t)dt\right)
\end{split}
\end{align*}


With that, we can apply the Laplace transform to $P$, $I$, and $D$.

% subsubsection: Laplace: Proportional
\subsubsection{Laplace: Proportional}

\begin{align*}
    \begin{split}
        P(s) &:= \mathcal{L}\{P(t)\} \\\\
             &= \int_{0}^{\infty} K_{p}e(t)e^{-st}dt \\\\
             &= K_{p}\int_{0}^{\infty}e(t)e^{-st}dt \\\\
             &= K_{p}\mathcal{L}\{e(t)\} \\\\
             &= K_{p}E(s)
    \end{split}
\end{align*}

% subsubsection: Laplace: Integral
\subsubsection{Laplace: Integral}

\begin{align*}
    \begin{split}
        I(s) &:= \mathcal{L}\{I(t)\} \\\\
             &= \int_{0}^{\infty}\left(K_{i}\int_{0}^{t}e(\tau)d\tau\right)e^{-st}dt \\\\
             &= K_{i}\int_{0}^{\infty}e^{-st}\int_{0}^{t}e(\tau)d\tau dt \\\\
             &= \left.K_{i}\left(\int_{0}^{t}e(\tau)d\tau \left(\frac{e^{-st}}{-s}\right)\right)\right|_{0}^{\infty} - \int_{0}^{\infty}e(t)\left(\frac{e^{-st}}{-s}\right)dt; \text{(integration by parts)} \\\\
             &= K_{i}\left(0 - 0 + \frac{1}{s}\int_{0}^{\infty}e(t)e^{-st}\right) \\\\
             &= K_{i}\left(\frac{\mathcal{L}\{e(t)\}}{s}\right) \\\\
             &= \frac{K_{i}E(s)}{s} 
    \end{split}
\end{align*}

% subsubsection: Laplace: Derivative
\subsubsection{Laplace: Derivative}

\begin{align*}
\begin{split}
    D(s) &:= \mathcal{L}\{D(t)\} \\\\
         &= \int_{0}^{\infty}K_{d}\frac{de(t)}{dt}e^{-st}dt \\\\
         &= K_{d} \int_{0}^{\infty}e^{-st}\frac{de(t)}{dt}dt \\\\
         &= K_{d}\left(e^{-st}e(t)\Bigr\rvert_{0}^{\infty} - \int_{0}^{\infty}-se^{st}e(t)dt\right) \\\\
         &= K_{d}\left(0 - e(0) + s \int_{0}^{\infty}e^{-st}e(t)dt\right) \\\\
         &= K_{d}\left(s\mathcal{L}\{e(t)\} - e(0)\right) \\\\
         &= K_{d}sE(s); \text{($e(0) = 0$, since no signal is supplied at $t = 0$).}
\end{split}
\end{align*}

$D(s)$ amplifies higher frequency measurements (or proces noise), resulting in large amounts of change in the output. To reduce the term's responsiveness to
high frequency noise, is it common to filter the measurements with a low-pass filter. One such implementation is to replace $D(t)$ with the following:

\[ D(t) = \frac{K_{d}sE(s)}{s\tau + 1},\] 

where $\tau \in \R$ is a relatively short contant of time.

% subsubsection: Laplace: u(t)
\subsubsection{Laplace: u(t)}

Applying the Laplace transform to \textbf{eq.\ref{eq1}} gives:

\begin{align*}
\begin{split}
    U(s) &= \mathcal{L}\{u(t)\} \\\\
         &= \mathcal{L}\{P(t) + I(t) + D(t)\} \\\\
         &= K_{p}E(s) + \frac{K_{i}E(s)}{s} + \frac{K_{d}sE(s)}{s\tau + 1}\\\\
         &= \left(K_{p} + \frac{K_{i}}{s} + \frac{K_{d}s}{s\tau + 1}\right)E(s)
\end{split}
\end{align*}

Dividing both sides by $E(s)$ gives the \textit{transfer function} $H_{a}(s)$, which in this case is the linear mapping of the Laplace transform of
the error signal, $E(s) = \mathcal{L}\{e(t)\}$, to the Laplace transform of the control variable, $U(s) = \mathcal{L}\{u(t)\}$ (with the filtered
derivative term):

\begin{equation} \label{eq3}
    \boxed{
        H_{a}(s) := \frac{U(s)}{E(s)} = K_{p} + \frac{K_{i}}{s} + \frac{K_{d}s}{s\tau + 1} 
    }
\end{equation}

% subsection: Bilinear Transform
\subsection{Bilinear Transform}

Before we derive a difference equation, we would first need to discretize the transfer function $H_{a}(s)$ in \textbf{eq.\ref{eq3}.} For our purposes, we will use
the bilinear transformation (also known as Tustin's method) for this, as it preserves stability by mapping the left half of the s-plane to the interior
of the unit circle in the z-domain. Furthermore, it avoids aliasing in the frequency response by mapping the entire imaginary axis fo the s-plane to the
unit circle.

In practice, this is done by substituting $s$ in $H_{a}(s)$ with the following polynomial:

\begin{equation} \label{eq4}
    \boxed{ 
    H(z) := \frac{U(z)}{E(z)} = H_{a}(s) \biggr\rvert_{s=\frac{2}{T}\frac{z-1}{z+1}} = H_{a}\left(\frac{2}{T}\frac{z-1}{z+1}\right)
    }
\end{equation}

As before, we divide \textbf{eq.\ref{eq3}} into three functions $P_{a}(s)$, $I_{a}(s)$, and $D_{a}(s)$, where:

\begin{align*}
\begin{split}
    P_{a}(s) &:= K_{p} \\
    I_{a}(s) &:= \frac{K_{i}}{s} \\
    D_{a}(s) &:= \frac{K_{d}s}{s\tau + 1} = \frac{K_{d}}{\tau + s^{-1}}
\end{split}
\end{align*}

Applying \textbf{eq.\ref{eq4}} to $P_{a}(s)$, $I_{a}(s)$, and $D_{a}(s)$ gives the transfer function $H(z)$.

% subsubsection: Bilinear: Proportional
\subsubsection{Bilinear: Proportional}

\begin{align*}
\begin{split}
    P_{b}(z) &:= P_{a}\left(\frac{2}{T}\frac{z-1}{z+1}\right) \\
             &= K_{p}
\end{split}
\end{align*}

% subsubsection: Bilinear: Integral
\subsubsection{Bilinear: Integral}

\begin{align*}
\begin{split}
    I_{b}(z) &:= I_{a}\left(\frac{2}{T}\frac{z-1}{z+1}\right) \\\\
             &= K_{i}\left(\frac{T}{2}\frac{z+1}{z-1}\right) \\\\
             &= K_{i}\left(\frac{T}{2}\frac{1+z^{-1}}{1-z^{-1}}\right)
\end{split}
\end{align*}

% subsubsection: Bilinear: Derivative
\subsubsection{Bilinear: Derivative}

\begin{align*}
\begin{split}
    D_{b}(z) &:= D_{a}\left(\frac{2}{T}\frac{z-1}{z+1}\right) \\\\
             &= K_{d}\left(\frac{1}{\tau + \frac{T}{2}\frac{z+1}{z-1}}\right) \\\\
             &= K_{d}\left(\frac{1}{\frac{2\tau (z-1) + T(z + 1)}{2(z-1)}}\right) \\\\
             &= K_{d}\left(\frac{2(z-1)}{2\tau (z-1) + T(z+1)}\right) \\\\
             &= K_{d}\left(\frac{2(1-z^{-1})}{2\tau (1-z^{-1}) + T(1+z^{-1})}\right) \\\\
\end{split}
\end{align*}

% subsubsection: Bilinear: Discrete Transfer Function
\subsubsection{Bilinear: Discrete Transfer Function}

Summing $P_{b}(z)$, $I_{b}(z)$, and $D_{b}(z)$ renders the transfer function for the PID algorithm in the z domain:

\begin{equation} \label{eq5}
\boxed{
    H_{b}(z) := \frac{U(z)}{E(z)} = P_{b}(z) + I_{b}(z) + D_{b}(z)
}
\end{equation}

% subsection: Difference Equation
\subsection{Difference Equation}

Converting \textbf{eq.\ref{eq5}} to an equivalent difference equation $u[n], n \in Z$ in the time domain can be accomplished by using the \textit{inverse
Z-transform}, denoted by $u[n] := \mathcal{Z}^{-1}\{U(z)\}$. To do this, we utilize a property of the Z-transform where taking the inverse
Z-transform of a function $X(z)$ multiplied by $z^{-k}$, $k > 0$, is equivalent to a shift by $k$ samples in the time domain, i.e.,
$\mathcal{Z}^{-1}\{z^{-1}X(z)\} = x[n-k].$

By implication, given two functions $Y(z)$ and $X(z)$, we can make the following observation (which will not be proven in this document in
the interest of brevity):
\begin{align*}
\begin{split}
    Y(z) = z^{-1}X(z) \Rightarrow y[n] = x[n-1].
\end{split}
\end{align*}

By multiplying both sides of \textbf{eq.\ref{eq5}} by $E(z)$ and taking the inverse Z-transform of the result, we get the following difference equation denoting
a discretized form of the PID algorithm:

\begin{equation} \label{eq6}
\boxed{
    u[n] := p[n] + i[n] + d[n],
}
\end{equation}

where, similar to $u(t)$ in \textbf{eq.\ref{eq1},} $u[n]$ is the \textit{control variable} at discrete time index $n$ and
\begin{align*}
\begin{split}
    p[n] &= K_{p}e[n] \\\\
    i[n] &= \frac{K_{i}T}{2}\left(e[n] + e[n-1]\right) + i[n-1] \\\\
    d[n] &= \frac{2K_{d}}{2\tau + T}\left(e[n] - e[n-1]\right) + \frac{2\tau - T}{2\tau + T}d[n-1].
\end{split}
\end{align*}

\newpage
Below is a block diagram of the discretized PID algorithm. Here, $r[n]$ denotes the \textit{setpoint} at time $n$ and $y[n]$ the \textit{process variable.}
Notice it is very similar to the block diagram for the continuous variant.

% Discrete PID block diagram
\begin{tikzpicture}[auto]
    % nodes: input
    \node [input, name=input]{}; 
    \node [sum, right = 1.3cm of input] (sum_1) {$\Sigma$};
    \node [blank, name = root, right = 1.7cm of sum_1]{};

    % nodes: PID
    \node [block, text width = 7em, right = 1cm of root] (pid_integral) {\textbf{I}: $i[n]$};
    \node [block, text width = 7em, above = of pid_integral] (pid_proportional) {\textbf{P}: $p[n]$};
    \node [block, text width = 7em, below = of pid_integral] (pid_derivative) {\textbf{D}: $d[n]$};

    % nodes: Output
    \node [sum, right = 2cm of pid_integral] (sum_2) {$\Sigma$}; 
    \node [block, right = 1.5cm of sum_2] (plant_process) {Plant Process};
    \node [blank, name=feedback, right = 0.2cm of plant_process]{};
    \node [output, name=output, right = 1.60cm of feedback]{};
    \node [block, below = of pid_derivative] (sensor) {Sensor};

    % arrows: input -> PID
    \draw [big_arrow] (input) -- node [above]{$r[n]$} node [below,pos=0.90] {$\arrowtext{+}$}(sum_1);
    \draw [-] (sum_1) -- node {$e[n]$}(root);
    \draw [big_arrow] (root) -- (pid_integral);
    \draw [big_arrow] (root) |- (pid_proportional);
    \draw [big_arrow] (root) |- (pid_derivative);

    % arrows: PID -> PV
    \draw [big_arrow] (pid_integral) -- node [pos=0.90]{$\arrowtext{+}$}(sum_2);
    \draw [big_arrow] (pid_proportional) -| node [pos=0.90]{$\arrowtext{+}$} (sum_2);
    \draw [big_arrow] (pid_derivative) -| node [pos=0.90]{$\arrowtext{+}$}(sum_2);
    \draw [big_arrow] (sum_2) -- node{$u[n]$}(plant_process);

    % arrows: p(v) -> r(t)
    \draw [-] (plant_process) -- (feedback);
    \draw [-] (feedback) |- (sensor);
    \draw [big_arrow] (sensor) -| node [below,pos=0.30] {measurement} node [pos=0.96]{$\arrowtext{-}$}(sum_1);
    \draw [big_arrow] (feedback) -- node {$y[n]$}(output);
\end{tikzpicture}

\end{document}
