% Page formatting
\documentclass[11pt]{article}
\usepackage{algorithm,algorithmic,amsmath,amsgen,amstext,amssymb,amsthm,amsbsy,amsopn,amsfonts,esint,hyperref}
\usepackage{bm}
\usepackage{tikz}
\usepackage[utf8]{inputenc}
\usepackage[english]{babel}
\usepackage{tabularx}
\pagestyle{plain}

\usetikzlibrary{shapes,arrows.meta,decorations.markings,positioning}

\setlength{\textwidth}{6.5 in}
\setlength{\textheight}{9 truein}
\setlength{\evensidemargin}{ 0 in}
\setlength{\oddsidemargin}{0 in}
\setlength{\topmargin}{-0.5 in}
\setlength{\parskip} {0.125 in}
\setlength{\parindent} {0 in}

\renewcommand{\familydefault}{\sfdefault}

% Remappings
\newcommand{\R}{\mathbb{R}}
\renewcommand\qedsymbol{ \blacksquare }
\newcommand{\arrowtext}[1]{\scriptstyle{\boldsymbol{#1}}}

% Theorems
\newtheorem{theorem}{Theorem}[section]
\newtheorem{corollary}{Corollary}[theorem]
\newtheorem{lemma}[theorem]{Lemma}

% Definitions
\theoremstyle{definition}
\newtheorem{definition}{Definition}[section]

% Beginning of document
\begin{document}
\noindent
\begin{minipage}{1.00\textwidth}
    \begin{flushleft}
        % title
        \textbf{\huge Project Report: EDMMS Temperature Controller} \\[1ex]
        % author
        \Large Jeremy Evans, Lorand Mezei, Macallister Armstrong, Anthony Kirkland
    \end{flushleft}
\noindent\rule{\textwidth}{1pt}
\end{minipage}

% tikz colors
\definecolor{wmugold}{RGB}{241,197,0}

% == tikzset ==
\tikzset{
    block/.style={
        draw,
        fill=wmugold,
        rectangle,
        minimum height=3em,
        minimum width=7em,
    },
    sum/.style={
        draw,
        fill=wmugold,
        circle,
        node distance=1cm
    },
    input/.style={coordinate},
    output/.style={coordinate},
    blank/.style={coordinate},
    pinstyle/.style={
        pin edge={to-,thin,black}
    },
    big_arrow/.style={
        decoration={markings,mark=at position 1 with {\arrow[scale=2,>=latex]{>}}},
        postaction={decorate},
        shorten >=0.4pt
    },
}

% =======================================================================================================================
% body

% section: Purpose
\section{Purpose}

This document introduces the theory of PID control and its programmatic implementation. We will define, at a high
level, what a PID controller is, what its purpose is, loop tuning, and its implementation in software. It is meant to
serve as an aid for designing the programming logic of a PID controller that utilizes an MSP430 MCU.

Portions of the following text may go into the final report upon approval.

% subsection: Version history
\subsection{Version history}

\begin{tabularx}{1.0\textwidth} {
    | >{\raggedright\arraybackslash}X
    | >{\raggedright\arraybackslash}X
    | >{\raggedright\arraybackslash}X
    | >{\raggedright\arraybackslash}X | }
    \hline
    \textbf{Date} & \textbf {Author} & \textbf{Comments} & \textbf {Version} \\
    \hline
    2/3/2021 & Jeremy Evans & Initial document & 1.0 \\
    \hline
\end{tabularx}

% section: Introduction
\section{Introduction}

A PID controller (proportional-integral-derivative controller) is a feedback control loop mechanism that receives a desired \textit{setpoint}
$r(t)$ as input and outputs a \textit{process variable} (PV), denoted as $y(t)$, which is then fed back into the system
as input. It continuously calculates an \textit{error value} $e(t)$ as the difference between the desired setpoint and the process variable
and applies a correction based on \textit{proportional} (P), \textit{integral} (I), and \textit{derivative} (D) terms, which are summed
together to make up the \textit{control variable}, $u(t)$ affecting the value of the process variable, hence the name. 
In applying this correction over time, it attempts to stabilize the output, i.e., eliminate the oscillation of the error value 
(or achieve marginal stability, or bounded oscillation, though specifications vary between applications).

Corrections are achieved by multiplying the P, I, and D terms by constants $K_{p}$, $K_{i}$, and $K_{d}$ respectively, each chosen through
a process known as \textit{loop tuning.} Improperly chosen constants would result in the error value diverging (with or without oscillation),
whereas properly chosen constants would have the desired effect of the error value converging.

\newpage

% subsection: Mathematical Definition
\subsection{Mathematical Definition}

PID control is mathematically defined as: 

\[ u(t) = K_{p}e(t) + K_{i}\int_{0}^{t}e(\tau)d(\tau) + K_{d}\frac{de(t)}{dt},\] 

where

\begin{itemize}
    \item $u(t)$ is the \textit{control variable}, the parameter that is controlled,
    \item $K_{p} \in \R \geq 0$ is the \textit{proportional gain,}
    \item $K_{i} \in \R \geq 0$ is the \textit{integral gain,}
    \item $K_{d} \in \R \geq 0$ is the \textit{derivative gain,}
    \item $e(t) = r(t) - y(t)$ is the error ($r(t) = SP$ is the \textit{setpoint,} $y(t) = PV$ is the \textit{process variable}),
    \item $t$ is the \textit{instantaneous time} (i.e., the current time),
    \item $\tau \in \R$ are measurements of time in the range $[0, t]$.
\end{itemize}

% subsection: PID Block Diagram
\subsection{PID Block Diagram}

The PID algorithm is given by the following block diagram:

% PID block diagram
\begin{tikzpicture}[auto]
    % nodes: input
    \node [input, name=input]{};
    \node [sum, right = 1.3cm of input] (sum_1) {$\Sigma$};
    \node [blank, name = root, right = 1.7cm of sum_1]{};
    
    % nodes: PID
    \node [block, text width = 7em, right = 1cm of root] (pid_integral) {\textbf{I}: $K_{i}\int_{0}^{t}e(\tau)d\tau$};
    \node [block, text width = 7em, above = of pid_integral] (pid_proportional) {\textbf{P}: $K_{p}e(t)$};
    \node [block, text width = 7em, below = of pid_integral] (pid_derivative) {\textbf{D}: $K_{d}\frac{de(t)}{dt}$};
    
    % nodes: Output
    \node [sum, right = 2cm of pid_integral] (sum_2) {$\Sigma$};
    \node [block, right = 1.5cm of sum_2] (plant_process) {Plant Process};
    \node [blank, name=feedback, right = 0.2cm of plant_process]{}; 
    \node [output, name=output, right = 1.60cm of feedback]{};
    \node [block, below = of pid_derivative] (sensor) {sensor};

    % arrows: input -> PID
    \draw [big_arrow] (input) -- node [above]{$r(t)$} node [below,pos=0.90] {$\arrowtext{+}$}(sum_1);
    \draw [-] (sum_1) -- node {$e(t)$}(root);
    \draw [big_arrow] (root) -- (pid_integral);
    \draw [big_arrow] (root) |- (pid_proportional);
    \draw [big_arrow] (root) |- (pid_derivative);

    % arrows: PID -> PV
    \draw [big_arrow] (pid_integral) -- node [pos=0.90]{$\arrowtext{+}$}(sum_2);
    \draw [big_arrow] (pid_proportional) -| node [pos=0.90]{$\arrowtext{+}$} (sum_2);
    \draw [big_arrow] (pid_derivative) -| node [pos=0.90]{$\arrowtext{+}$}(sum_2);
    \draw [big_arrow] (sum_2) -- node{$u(t)$}(plant_process);

    % arrows: p(v) -> r(t)
    \draw [-] (plant_process) -- (feedback);
    \draw [-] (feedback) |- (sensor);
    \draw [big_arrow] (sensor) -| node [below,pos=0.30] {measurement} node [pos=0.96]{$\arrowtext{-}$}(sum_1);
    \draw [big_arrow] (feedback) -- node {$y(t)$}(output);
\end{tikzpicture}

Represented in a high-level model of a heat treatment furnace, $r(t),$ the setpoint, would be the desired
furnace temperature; $u(t),$ the control variable, would be the gas flow rate; and $y(t),$ the process
variable, would be the measured furnace temperature.

% subsection: PID Terms
\subsection{PID Terms}

% subsubsection: Proportional
\subsubsection{Proportional}

Given by $P_{out} = K_{p}e(t),$ the proportional term produces an output that is proportional to the current error value $e(t);$ the greater
the error value, the greater the control output. This output is multiplied by a gain factor $K_{p} \in \R \geq 0$ that determines how responsive
the controller should be with a given error-value. Large values of $K_{p}$ result in a large change in the output for a given change in the
error. However, exceptionally large values may result in the output \textit{overshooting} the setpoint (exceeding the value of the setpoint)
and, in the worst case, destabilizing the system whereupon the error rate diverges.

% subsubsection: Integral
\subsubsection{Integral}

Given by $I_{out} = K_{i}\int_{0}^{t}e(\tau)d\tau$

% subsubsection: Derivative
\subsubsection{Derivative}

Given by $K_{i}\int_{0}^{t}e(\tau)d\tau$

% subsection: Loop Tuning
\subsection{Loop Tuning}

\end{document}
